\documentclass[pdf]
{beamer}
\mode<presentation>{}
%% preamble
\usetheme{Warsaw}

%% preamble
\title{CUR Decomposition and Its Applications}
\subtitle{A Comprehensive Overview}
\author{Kevin Smith}
\date{\today}




\begin{document}

\begin{frame}
    \titlepage
\end{frame}

\begin{frame}{Introduction}
    \begin{itemize}
        \item \textbf{Brief overview of matrix factorizations:}
            \begin{itemize}
                \item Matrix factorizations are fundamental techniques in linear algebra used to decompose matrices into products of simpler matrices.
                \item Common types include LU, QR, and Singular Value Decomposition (SVD), each serving different purposes in numerical analysis, optimization, and data science.
            \end{itemize}
        \item \textbf{Introduction to CUR decomposition:}
            \begin{itemize}
                \item Unlike traditional factorizations, CUR decomposition selects actual columns and rows from the original matrix to form matrices C and R, with a middle matrix U to link them.
                \item This method is particularly valuable for large sparse datasets where interpretability of the factors is crucial.
            \end{itemize}
        \item \textbf{Importance and advantages in data analysis:}
            \begin{itemize}
                \item CUR decomposition provides a more interpretable and often more efficient alternative to SVD for approximating matrices.
                \item It's especially useful in areas like image processing, recommender systems, and bioinformatics, where understanding the significance of the data's features and observations directly matters.
            \end{itemize}
    \end{itemize}
\end{frame}

\begin{frame}{Moore-Penrose Pseudoinverse}
    \begin{definition}
        The Moore-Penrose pseudoinverse of a matrix \( A \) is defined as the matrix \( A^+ \) that satisfies the following conditions:
        \begin{enumerate}
            \item \( AA^+A = A \) \quad (Reproduction of \( A \))
            \item \( A^+AA^+ = A^+ \) \quad (Reproduction of \( A^+ \))
            \item \( (AA^+)^T = AA^+ \) \quad (Hermitian property of \( AA^+ \))
            \item \( (A^+A)^T = A^+A \) \quad (Hermitian property of \( A^+A \))
        \end{enumerate}
    \end{definition}
\end{frame}


\begin{frame}{CUR Background}
    \begin{itemize}
        \item \textbf{What is CUR Decomposition?}
            \begin{itemize}
                \item CUR decomposition approximates a matrix \(A\) using selected columns \(C\) and rows \(R\) from \(A\), combined through a middle matrix \(U\) to approximate \(A\) as \(CUR\).
            \end{itemize}
        \item \textbf{Mathematical formulation:}
            \begin{itemize}
                \item Given \(A \in \mathbb{R}^{m \times n}\), select subsets of columns and rows to form \(C\) and \(R\).
                \item Compute \(U\) as \(U = C^+ A R^+\), where \(C^+\) and \(R^+\) are Moore-Penrose pseudoinverses, minimizing the error \( \| A - CUR \|_F \).
            \end{itemize}
        \item \textbf{Comparison with SVD:}
            \begin{itemize}
                \item SVD decomposes \(A\) into \(A = U \Sigma V^T\) with optimal low-rank approximation but uses abstract, non-intuitive singular vectors and values.
                \item In data analysis, interpretability is key; CUR's use of actual data columns and rows enhances understandability and relevance in applied settings, making it superior for tasks requiring clear, actionable insights.
            \end{itemize}
    \end{itemize}
\end{frame}



\begin{frame}{Algorithm}
    \begin{itemize}
        \item How CUR Decomposition Works
        \item Selection criteria for columns (C) and rows (R)
        \item Practical implementation steps
    \end{itemize}
\end{frame}

\begin{frame}{Advantages of CUR}
    \begin{itemize}
        \item Interpretability of the components
        \item Computational benefits
        \item Application contexts where CUR excels
    \end{itemize}
\end{frame}

\begin{frame}{Application in Data Analysis}
    \begin{itemize}
        \item Overview of different applications
        \item Highlight on key use cases
    \end{itemize}
\end{frame}

\begin{frame}{Image Compression}
    \begin{itemize}
        \item Using CUR for image compression
        \item Example with results
    \end{itemize}
\end{frame}

\begin{frame}{Recommender Systems}
    \begin{itemize}
        \item Application in collaborative filtering
        \item Benefits over other matrix factorizations
    \end{itemize}
\end{frame}

\begin{frame}{Gene Expression}
    \begin{itemize}
        \item CUR in bioinformatics
        \item Case study: Identifying significant genes
    \end{itemize}
\end{frame}

\begin{frame}{Dimensionality Reduction}
    \begin{itemize}
        \item CUR vs PCA in feature selection
        \item Advantages in interpretability and selection
    \end{itemize}
\end{frame}

\begin{frame}{Potential Drawbacks}
    \begin{itemize}
        \item Limitations of CUR decomposition
        \item Conditions for optimal performance
    \end{itemize}
\end{frame}

\begin{frame}{Q\&A}
    \begin{itemize}
        \item Invitation for audience to ask questions or discuss further
    \end{itemize}
\end{frame}

\end{document}
